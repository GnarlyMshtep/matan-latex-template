% matan includes
\tcbuselibrary{most}
\usepackage{outlines}				%out enviroment 
\usepackage{xspace}                 %so we don't need
\usepackage{cancel}                 %\not across multiple symbols %paths are relative to main.tex! 
%! avoiding using these in macros so they could be turned off when the override another macro

\newcommand{\cA}{\ensuremath{\mathcal{A}}\xspace}
\newcommand{\cB}{\ensuremath{\mathcal{B}}\xspace}
\newcommand{\cC}{\ensuremath{\mathcal{C}}\xspace}
\newcommand{\cD}{\ensuremath{\mathcal{D}}}
\newcommand{\cE}{\ensuremath{\mathcal{E}}\xspace}
\newcommand{\cF}{\ensuremath{\mathcal{F}}\xspace}
\newcommand{\cG}{\ensuremath{\mathcal{G}}\xspace}
\newcommand{\cH}{\ensuremath{\mathcal{H}}\xspace}
\newcommand{\cI}{\ensuremath{\mathcal{I}}\xspace}
\newcommand{\cJ}{\ensuremath{\mathcal{J}}\xspace}
\newcommand{\cK}{\ensuremath{\mathcal{K}}\xspace}
\newcommand{\cL}{\ensuremath{\mathcal{L}}\xspace}
\newcommand{\cM}{\ensuremath{\mathcal{M}}\xspace}
\newcommand{\cN}{\ensuremath{\mathcal{N}}\xspace}
\newcommand{\cO}{\ensuremath{\mathcal{O}}\xspace}
\newcommand{\cP}{\ensuremath{\mathcal{P}}\xspace}
\newcommand{\cQ}{\ensuremath{\mathcal{Q}}\xspace}
\newcommand{\cR}{\ensuremath{\mathcal{R}}\xspace}
\newcommand{\cS}{\ensuremath{\mathcal{S}}\xspace}
\newcommand{\cT}{\ensuremath{\mathcal{T}}\xspace}
\newcommand{\cU}{\ensuremath{\mathcal{U}}\xspace}
\newcommand{\cV}{\ensuremath{\mathcal{V}}\xspace}
\newcommand{\cW}{\ensuremath{\mathcal{W}}\xspace}
\newcommand{\cX}{\ensuremath{\mathcal{X}}\xspace}
\newcommand{\cY}{\ensuremath{\mathcal{Y}}\xspace}
\newcommand{\cZ}{\ensuremath{\mathcal{Z}}\xspace}

\newcommand{\fA}{\ensuremath{\mathfrak{A}}\xspace}
\newcommand{\fB}{\ensuremath{\mathfrak{B}}\xspace}
\newcommand{\fC}{\ensuremath{\mathfrak{C}}\xspace}
\newcommand{\fD}{\ensuremath{\mathfrak{D}}\xspace}
\newcommand{\fE}{\ensuremath{\mathfrak{E}}\xspace}
\newcommand{\fF}{\ensuremath{\mathfrak{F}}\xspace}
\newcommand{\fG}{\ensuremath{\mathfrak{G}}\xspace}
\newcommand{\fH}{\ensuremath{\mathfrak{H}}\xspace}
\newcommand{\fI}{\ensuremath{\mathfrak{I}}\xspace}
\newcommand{\fJ}{\ensuremath{\mathfrak{J}}\xspace}
\newcommand{\fK}{\ensuremath{\mathfrak{K}}\xspace}
\newcommand{\fL}{\ensuremath{\mathfrak{L}}\xspace}
\newcommand{\fM}{\ensuremath{\mathfrak{M}}\xspace}
\newcommand{\fN}{\ensuremath{\mathfrak{N}}\xspace}
\newcommand{\fO}{\ensuremath{\mathfrak{O}}\xspace}
\newcommand{\fP}{\ensuremath{\mathfrak{P}}\xspace}
\newcommand{\fQ}{\ensuremath{\mathfrak{Q}}\xspace}
\newcommand{\fR}{\ensuremath{\mathfrak{R}}\xspace}
\newcommand{\fS}{\ensuremath{\mathfrak{S}}\xspace}
\newcommand{\fT}{\ensuremath{\mathfrak{T}}\xspace}
\newcommand{\fU}{\ensuremath{\mathfrak{U}}\xspace}
\newcommand{\fV}{\ensuremath{\mathfrak{V}}\xspace}
\newcommand{\fW}{\ensuremath{\mathfrak{W}}\xspace}
\newcommand{\fX}{\ensuremath{\mathfrak{X}}\xspace}
\newcommand{\fY}{\ensuremath{\mathfrak{Y}}\xspace}
\newcommand{\fZ}{\ensuremath{\mathfrak{Z}}\xspace}

\newcommand{\fa}{\ensuremath{\mathfrak{a}}\xspace}
\newcommand{\fb}{\ensuremath{\mathfrak{b}}\xspace}
\newcommand{\fc}{\ensuremath{\mathfrak{c}}\xspace}
\newcommand{\fd}{\ensuremath{\mathfrak{d}}\xspace}
\newcommand{\fe}{\ensuremath{\mathfrak{e}}\xspace}
\newcommand{\fg}{\ensuremath{\mathfrak{g}}\xspace}
\newcommand{\fh}{\ensuremath{\mathfrak{h}}\xspace}
% \newcommand{\fi}{\ensuremath{\mathfrak{i}}\xspace} end if statement 
\newcommand{\fj}{\ensuremath{\mathfrak{j}}\xspace}
\newcommand{\fk}{\ensuremath{\mathfrak{k}}\xspace}
\newcommand{\fl}{\ensuremath{\mathfrak{l}}\xspace}
\newcommand{\fm}{\ensuremath{\mathfrak{m}}\xspace}
\newcommand{\fn}{\ensuremath{\mathfrak{n}}\xspace}
\newcommand{\fo}{\ensuremath{\mathfrak{o}}\xspace}
\newcommand{\fp}{\ensuremath{\mathfrak{p}}\xspace}
\newcommand{\fq}{\ensuremath{\mathfrak{q}}\xspace}
\newcommand{\fr}{\ensuremath{\mathfrak{r}}\xspace}
\newcommand{\fs}{\ensuremath{\mathfrak{s}}\xspace}
\newcommand{\ft}{\ensuremath{\mathfrak{t}}\xspace}
\newcommand{\fu}{\ensuremath{\mathfrak{u}}\xspace}
\newcommand{\fv}{\ensuremath{\mathfrak{v}}\xspace}
\newcommand{\fw}{\ensuremath{\mathfrak{w}}\xspace}
\newcommand{\fx}{\ensuremath{\mathfrak{x}}\xspace}
\newcommand{\fy}{\ensuremath{\mathfrak{y}}\xspace}
\newcommand{\fz}{\ensuremath{\mathfrak{z}}\xspace}

\newcommand{\bA}{\ensuremath{\bm{A}}\xspace}
\newcommand{\bB}{\ensuremath{\bm{B}}\xspace}
\newcommand{\bC}{\ensuremath{\bm{C}}\xspace}
\newcommand{\bD}{\ensuremath{\bm{D}}\xspace}
\newcommand{\bE}{\ensuremath{\bm{E}}\xspace}
\newcommand{\bF}{\ensuremath{\bm{F}}\xspace}
\newcommand{\bG}{\ensuremath{\bm{G}}\xspace}
\newcommand{\bH}{\ensuremath{\bm{H}}\xspace}
\newcommand{\bI}{\ensuremath{\bm{I}}\xspace}
\newcommand{\bJ}{\ensuremath{\bm{J}}\xspace}
\newcommand{\bK}{\ensuremath{\bm{K}}\xspace}
\newcommand{\bL}{\ensuremath{\bm{L}}\xspace}
\newcommand{\bM}{\ensuremath{\bm{M}}\xspace}
\newcommand{\bN}{\ensuremath{\bm{N}}\xspace}
\newcommand{\bO}{\ensuremath{\bm{O}}\xspace}
\newcommand{\bP}{\ensuremath{\bm{P}}\xspace}
\newcommand{\bQ}{\ensuremath{\bm{Q}}\xspace}
\newcommand{\bR}{\ensuremath{\bm{R}}\xspace}
\newcommand{\bS}{\ensuremath{\bm{S}}\xspace}
\newcommand{\bT}{\ensuremath{\bm{T}}\xspace}
\newcommand{\bU}{\ensuremath{\bm{U}}\xspace}
\newcommand{\bV}{\ensuremath{\bm{V}}\xspace}
\newcommand{\bW}{\ensuremath{\bm{W}}\xspace}
\newcommand{\bX}{\ensuremath{\bm{X}}\xspace}
\newcommand{\bY}{\ensuremath{\bm{Y}}\xspace}
\newcommand{\bZ}{\ensuremath{\bm{Z}}\xspace}

\newcommand{\sA}{\ensuremath{\mathsf{A}}\xspace}
\newcommand{\sB}{\ensuremath{\mathsf{B}}\xspace}
\newcommand{\sC}{\ensuremath{\mathsf{C}}\xspace}
\newcommand{\sD}{\ensuremath{\mathsf{D}}\xspace}
\newcommand{\sE}{\ensuremath{\mathsf{E}}\xspace}
\newcommand{\sF}{\ensuremath{\mathsf{F}}\xspace}
\newcommand{\sG}{\ensuremath{\mathsf{G}}\xspace}
\newcommand{\sH}{\ensuremath{\mathsf{H}}\xspace}
\newcommand{\sI}{\ensuremath{\mathsf{I}}\xspace}
\newcommand{\sJ}{\ensuremath{\mathsf{J}}\xspace}
\newcommand{\sK}{\ensuremath{\mathsf{K}}\xspace}
\newcommand{\sL}{\ensuremath{\mathsf{L}}\xspace}
\newcommand{\sM}{\ensuremath{\mathsf{M}}\xspace}
\newcommand{\sN}{\ensuremath{\mathsf{N}}\xspace}
\newcommand{\sO}{\ensuremath{\mathsf{O}}\xspace}
\newcommand{\sP}{\ensuremath{\mathsf{P}}\xspace}
\newcommand{\sQ}{\ensuremath{\mathsf{Q}}\xspace}
\newcommand{\sR}{\ensuremath{\mathsf{R}}\xspace}
\newcommand{\sS}{\ensuremath{\mathsf{S}}\xspace}
\newcommand{\sT}{\ensuremath{\mathsf{T}}\xspace}
\newcommand{\sU}{\ensuremath{\mathsf{U}}\xspace}
\newcommand{\sV}{\ensuremath{\mathsf{V}}\xspace}
\newcommand{\sW}{\ensuremath{\mathsf{W}}\xspace}
\newcommand{\sX}{\ensuremath{\mathsf{X}}\xspace}
\newcommand{\sY}{\ensuremath{\mathsf{Y}}\xspace}
\newcommand{\sZ}{\ensuremath{\mathsf{Z}}\xspace}

% \newcommand{\ba}{\ensuremath{\bm{a}}\xspace}
% \newcommand{\bb}{\ensuremath{\bm{b}}\xspace}
% \newcommand{\bc}{\ensuremath{\bm{c}}\xspace}
% \newcommand{\bd}{\ensuremath{\bm{d}}\xspace}
% \newcommand{\be}{\ensuremath{\bm{e}}\xspace}
% % \newcommand{\bf}{\ensuremath{\bm{f}}\xspace} taken for boldface
% \newcommand{\bg}{\ensuremath{\bm{g}}\xspace}
% \newcommand{\bh}{\ensuremath{\bm{h}}\xspace}
% \newcommand{\bi}{\ensuremath{\bm{i}}\xspace}
% \newcommand{\bj}{\ensuremath{\bm{j}}\xspace}
% \newcommand{\bk}{\ensuremath{\bm{k}}\xspace}
% \newcommand{\bl}{\ensuremath{\bm{l}}\xspace}
% %\newcommand{\bm}{\ensuremath{\bm{m}}\xspace} taken for bold math
% \newcommand{\bn}{\ensuremath{\bm{n}}\xspace}
% \newcommand{\bo}{\ensuremath{\bm{o}}\xspace}
% \newcommand{\bp}{\ensuremath{\bm{p}}\xspace}
% \newcommand{\bq}{\ensuremath{\bm{q}}\xspace}
% \newcommand{\br}{\ensuremath{\bm{r}}\xspace}
% \newcommand{\bs}{\ensuremath{\bm{s}}\xspace}
% \newcommand{\bt}{\ensuremath{\bm{t}}\xspace}
% \newcommand{\bu}{\ensuremath{\bm{u}}\xspace}
% \newcommand{\bv}{\ensuremath{\bm{v}}\xspace}
% \newcommand{\bw}{\ensuremath{\bm{w}}\xspace}
% \newcommand{\bx}{\ensuremath{\bm{x}}\xspace}
% \newcommand{\by}{\ensuremath{\bm{y}}\xspace}
% \newcommand{\bz}{\ensuremath{\bm{z}}\xspace}

\newcommand{\bfA}{\ensuremath{\mathbf{A}}\xspace}
\newcommand{\bfB}{\ensuremath{\mathbf{B}}\xspace}
\newcommand{\bfC}{\ensuremath{\mathbf{C}}\xspace}
\newcommand{\bfD}{\ensuremath{\mathbf{D}}\xspace}
\newcommand{\bfE}{\ensuremath{\mathbf{E}}\xspace}
\newcommand{\bfF}{\ensuremath{\mathbf{F}}\xspace}
\newcommand{\bfG}{\ensuremath{\mathbf{G}}\xspace}
\newcommand{\bfH}{\ensuremath{\mathbf{H}}\xspace}
\newcommand{\bfI}{\ensuremath{\mathbf{I}}\xspace}
\newcommand{\bfJ}{\ensuremath{\mathbf{J}}\xspace}
\newcommand{\bfK}{\ensuremath{\mathbf{K}}\xspace}
\newcommand{\bfL}{\ensuremath{\mathbf{L}}\xspace}
\newcommand{\bfM}{\ensuremath{\mathbf{M}}\xspace}
\newcommand{\bfN}{\ensuremath{\mathbf{N}}\xspace}
\newcommand{\bfO}{\ensuremath{\mathbf{O}}\xspace}
\newcommand{\bfP}{\ensuremath{\mathbf{P}}\xspace}
\newcommand{\bfQ}{\ensuremath{\mathbf{Q}}\xspace}
\newcommand{\bfR}{\ensuremath{\mathbf{R}}\xspace}
\newcommand{\bfS}{\ensuremath{\mathbf{S}}\xspace}
\newcommand{\bfT}{\ensuremath{\mathbf{T}}\xspace}
\newcommand{\bfU}{\ensuremath{\mathbf{U}}\xspace}
\newcommand{\bfV}{\ensuremath{\mathbf{V}}\xspace}
\newcommand{\bfW}{\ensuremath{\mathbf{W}}\xspace}
\newcommand{\bfX}{\ensuremath{\mathbf{X}}\xspace}
\newcommand{\bfY}{\ensuremath{\mathbf{Y}}\xspace}
\newcommand{\bfZ}{\ensuremath{\mathbf{Z}}\xspace}

\newcommand{\barA}{\ensuremath{\bar{A}}\xspace}
\newcommand{\barB}{\ensuremath{\bar{B}}\xspace}
\newcommand{\barC}{\ensuremath{\bar{C}}\xspace}
\newcommand{\barD}{\ensuremath{\bar{D}}\xspace}
\newcommand{\barE}{\ensuremath{\bar{E}}\xspace}
\newcommand{\barF}{\ensuremath{\bar{F}}\xspace}
\newcommand{\barG}{\ensuremath{\bar{G}}\xspace}
\newcommand{\barH}{\ensuremath{\bar{H}}\xspace}
\newcommand{\barI}{\ensuremath{\bar{I}}\xspace}
\newcommand{\barJ}{\ensuremath{\bar{J}}\xspace}
\newcommand{\barK}{\ensuremath{\bar{K}}\xspace}
\newcommand{\barL}{\ensuremath{\bar{L}}\xspace}
\newcommand{\barM}{\ensuremath{\bar{M}}\xspace}
\newcommand{\barN}{\ensuremath{\bar{N}}\xspace}
\newcommand{\barO}{\ensuremath{\bar{O}}\xspace}
\newcommand{\barP}{\ensuremath{\bar{P}}\xspace}
\newcommand{\barQ}{\ensuremath{\bar{Q}}\xspace}
\newcommand{\barR}{\ensuremath{\bar{R}}\xspace}
\newcommand{\barS}{\ensuremath{\bar{S}}\xspace}
\newcommand{\barT}{\ensuremath{\bar{T}}\xspace}
\newcommand{\barU}{\ensuremath{\bar{U}}\xspace}
\newcommand{\barV}{\ensuremath{\bar{V}}\xspace}
\newcommand{\barW}{\ensuremath{\bar{W}}\xspace}
\newcommand{\barX}{\ensuremath{\bar{X}}\xspace}
\newcommand{\barY}{\ensuremath{\bar{Y}}\xspace}
\newcommand{\barZ}{\ensuremath{\bar{Z}}\xspace}

\newcommand{\bara}{\ensuremath{\bar{a}}\xspace}
\newcommand{\barb}{\ensuremath{\bar{b}}\xspace}
\newcommand{\barc}{\ensuremath{\bar{c}}\xspace}
\newcommand{\bard}{\ensuremath{\bar{d}}\xspace}
\newcommand{\bare}{\ensuremath{\bar{e}}\xspace}
\newcommand{\barf}{\ensuremath{\bar{f}}\xspace}
\newcommand{\barg}{\ensuremath{\bar{g}}\xspace}
\newcommand{\barh}{\ensuremath{\bar{h}}\xspace}
\newcommand{\bari}{\ensuremath{\bar{i}}\xspace}
\newcommand{\barj}{\ensuremath{\bar{j}}\xspace}
\newcommand{\bark}{\ensuremath{\bar{k}}\xspace}
\newcommand{\barl}{\ensuremath{\bar{l}}\xspace}
\newcommand{\barm}{\ensuremath{\bar{m}}\xspace}
\newcommand{\barn}{\ensuremath{\bar{n}}\xspace}
%\newcommand{\baro}{\ensuremath{\bar{o}}\xspace}
\newcommand{\barp}{\ensuremath{\bar{p}}\xspace}
\newcommand{\barq}{\ensuremath{\bar{q}}\xspace}
\newcommand{\barr}{\ensuremath{\bar{r}}\xspace}
\newcommand{\bars}{\ensuremath{\bar{s}}\xspace}
\newcommand{\bart}{\ensuremath{\bar{t}}\xspace}
\newcommand{\baru}{\ensuremath{\bar{u}}\xspace}
\newcommand{\barv}{\ensuremath{\bar{v}}\xspace}
\newcommand{\barw}{\ensuremath{\bar{w}}\xspace}
\newcommand{\barx}{\ensuremath{\bar{x}}\xspace}
\newcommand{\bary}{\ensuremath{\bar{y}}\xspace}
\newcommand{\barz}{\ensuremath{\bar{z}}\xspace}
% \input{lib/snark-shortcuts}
% add your macros here. Try to seperate them by categories, for example

%%%%%%%%%%%%%%%%%%%%%%%%%%%%%%%%%%%%%%%%%%%%%%
%%%%%%%%%%%%%%%%%%%%%%%%%%%%%%%%%%%%%%%%%%%%%%
%%%% text macros



%%%%%%%%%%%%%%%%%%%%%%%%%%%%%%%%%%%%%%%%%%%%%%%%%%%%%%%%%%%%%
%%%%%%%%%%%%%%%%%%%%%%%%%%%%%%%%%%%%%%%%%%%%%%%%%%%%%%%%%%%%%
% Envieroments

\newmdenv[
    topline=false,
    bottomline=false,
    rightline=false,
    leftline=true, % Only the left line is enabled
    linecolor=gray!40, % Light gray color for the line
    linewidth=4pt, % Width of the line
    leftmargin=5pt, % Indent from the left margin
    rightmargin=10pt, % Right margin spacing
    innertopmargin=5pt, % Space above the text inside the frame
    innerbottommargin=5pt % Space below the text inside the frame
]{indentedtext}

%! status does not play nice with flot yet
\newmdenv[
    topline=false,
    bottomline=true,
    rightline=false,
    leftline=true, % Only the left line is enabled
    linecolor=purple!40, % Light purple color for the line
    linewidth=4pt, % Width of the line
    leftmargin=5pt, % Indent from the left margin
    rightmargin=10pt, % Right margin spacing
    innertopmargin=5pt, % Space above the text inside the frame
    innerbottommargin=10pt, % Space below the text inside the frame
    roundcorner=10pt, % Rounded corners
]{statusReportInternal}

\newcommand{\status}[2]{
    \begin{statusReportInternal}
        \parhead{Status}
        \emph{#1}\\
        \par\noindent #2
    \end{statusReportInternal}
}

\newenvironment{out}
{\begin{outline}[enumerate]
}
{ 
    \end{outline}
}

\newenvironment{researchquestion}
{
    \itshape
    \begin{center}
    \vspace{.1em}
}
{
    \end{center}
    \vspace{.1em}
}

\NewDocumentCommand{\largeProb}{m m m O{=}}{
    \begin{equation*}
        \Pr\insqr{#1 \;\left| \;  
        \begin{aligned}
            #2 
        \end{aligned}
        \right.} #4 #3
    \end{equation*}
    }
    
\newcommand{\diffLargeProb}[5]{
\begin{dmath*}
\inabs{
    \Pr \insqr{ #1 \; \left| \;
    \begin{aligned}
        #2
    \end{aligned}
    }\right.
    -
    \Pr \insqr{ #3 \; \left| \;
    \begin{aligned}
        #4
    \end{aligned}
    }\right.
    } = #5	
\end{dmath*}
}

\newcommand{\eqLargeDist}[5]{
    \begin{dmath*}
        \inset{ #1 \; \left| \;
        \begin{aligned}
            #2
        \end{aligned}
        }\right.
        #3
        \inset{ #4 \; \left| \;
        \begin{aligned}
            #5
        \end{aligned}
        }\right.	
    \end{dmath*}
    }

%%%%%%%%%%%%%%%%%%%%%%%%%%%%%%%%%%%%%%%%%%%%%%%%%%%%%%%%%%%%%
%%%%%%%%%%%%%%%%%%%%%%%%%%%%%%%%%%%%%%%%%%%%%%%%%%%%%%%%%%%%%
%%%% in*  
\newcommand{\inpar}[1]{\ensuremath{\left( #1 \right)}\xspace}
\newcommand{\insqr}[1]{\ensuremath{\left[ #1 \right]}\xspace}
\newcommand{\inabs}[1]{\ensuremath{\left| #1 \right|}\xspace}
\newcommand{\inset}[1]{\ensuremath{\left\{ #1 \right\}}\xspace}
\newcommand{\inceil}[1]{\ensuremath{\left\lceil #1 \right\rceil}\xspace}
\newcommand{\infloor}[1]{\ensuremath{\left\lfloor #1 \right\rfloor}\xspace}


%%%%%%%%%%%%%%%%%%%%%%%%%%%%%%%%%%%%%%%%%%%%%%%%%%%%%%%%%%%%%
%%%%%%%%%%%%%%%%%%%%%%%%%%%%%%%%%%%%%%%%%%%%%%%%%%%%%%%%%%%%%
% General Math 

%markdown
\newcommand{\unestablished}{\XXX[unestablished]}


%symbols
\renewcommand{\phi}{\varphi}
\newcommand{\eps}{\ensuremath{\epsilon}\xspace}
\renewcommand{\emptyset}{\varnothing}
\newcommand{\Z}{\ensuremath{\mathbb{Z}}}
\newcommand{\ZZ}[1]{\ensuremath{\Z_{#1}}}
\newcommand{\vv}[1]{\ensuremath{\vec{#1}}\xspace}
\newcommand{\qeq}{{\stackrel{?}{=}}}
\newcommand{\tensor}{\otimes}
%
\newcommand{\GeneratedBy}[1]{\ensuremath{\langle #1 \rangle}\xspace}
\newcommand{\convolution}{\ensuremath{*}\xspace}
\newcommand{\divides}{\ensuremath{\:|\:}\xspace}

%decorators 
\newcommand{\inv}{\ensuremath{{^{-1}}}\xspace}
\renewcommand{\sc}[1]{^{(#1)}}
\newcommand{\Support}{{\mathrm{Support}}}
\newcommand{\norm}[1]{\left\lVert#1\right\rVert}


%% graph theory
\newcommand{\Graph}{G}
\newcommand{\Nei}{\ensuremath{\mathrm{Nbr}}}
\newcommand{\eigen}{\ensuremath{\lambda}\xspace}
\newcommand{\Cayley}[2]{\mathrm{Cay}(#1, #2)}
\newcommand{\CayleySquared}[3]{\mathrm{Cay}^2(#1, #2, #3)}


%indexes
\newcommand{\istar}{\ensuremath{i^\star}\xspace}
\newcommand{\jstar}{\ensuremath{j^\star}\xspace}

%styling
\newcommand{\Greymidrule}{\arrayrulecolor{lightgray}\midrule\arrayrulecolor{black}}
\newcommand{\horizLine}{\noindent\textcolor[RGB]{220,220,220}{\rule{\linewidth}{2pt}}}
\newcommand{\suchthat}{\ensuremath{~\middle|~}}
\newcommand{\defined}{\ensuremath{:=}}

% Lin Algebra 
\newcommand{\spanLA}{\ensuremath{\mathrm{span}}\xspace}


% other math 
\newcommand{\perm}{\pi}


